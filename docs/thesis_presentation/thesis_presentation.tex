\documentclass{beamer}
%\documentclass[aspectratio=169]{beamer}  % inna proporcja slajdu

\usepackage[utf8]{inputenc}
\usepackage{polski}

\usetheme{pwrlite}
%\usetheme[nosections]{pwrlite}  % wyłączenie stron z sekcjami
%\usetheme[en]{pwrlite}  % logo w wersji angielskiej
%\usetheme[nosections,en]{pwrlite}  % logo w wersji angielskiej i bez sekcji

%\pdfmapfile{+lato.map}  % jeśli nie działa font Lato (a jest zainstalowany), odkomentuj przy pierwszej kompilacji

\title{\large Wykrywanie oszustw na kartach płatniczych z~wykorzystaniem metod wrażliwych na koszt}
\institute{Wydział Matematyki Politechniki Wrocławskiej \\ Promotor: dr inż. Andrzej Giniewicz}
\author{Patryk Wielopolski}
\date{06-02-2020}

\begin{document}

\section{Wstęp}

\begin{frame}{Problem detekcji oszustw}
	Możliwości popełnienia przestępstwa
\end{frame}

\begin{frame}{Schemat pracy systemu}
	Rysunek
\end{frame}

\begin{frame}{Oznaczenia}
	Oznaczenia
\end{frame}

\section{Część teoretyczna}

\begin{frame}{Miary skuteczności modeli, cz. I}
	Macierz pomyłek
	Precyzja
	Czułość
	F1 Score
\end{frame}

\begin{frame}{Miary skuteczności modeli, cz. II}
	Koszt całkowity
	Oszczędności
\end{frame}

\begin{frame}{Standardowe modele predykcyjne}
	\begin{itemize}
		\item Regresja logistyczna
		\item Drzewo decyzyjne
		\item Las losowy
		\item XGBoost
	\end{itemize}
\end{frame}

\begin{frame}{Klasyfikacja wrażliwa na koszt}
	BMR
	TO
\end{frame}

\begin{frame}
	Regresja logistyczna wrażliwa na koszt
	Drzewo decyzyjne wrażliwe na koszt
\end{frame}


\section{Wyniki}

\begin{frame}{Oszczędności}
	Oszczędności
\end{frame}

\begin{frame}{F1 Score}
	F1 Score
\end{frame}

\begin{frame}{Wnioski}
	Wnioski
\end{frame}

\end{document}

