\documentclass{book}

\usepackage{amsmath}
\usepackage{listings}

\usepackage{booktabs}
\usepackage{graphicx}
\usepackage{multirow}
\usepackage{makecell}
\setcellgapes{5pt}

\usepackage[utf8]{inputenc}
\usepackage{polski}

\title{Detekcja oszustw z wykorzystaniem metod wrażliwych na koszt}
\author{Patryk Wielopolski}

\begin{document}

\maketitle

\chapter{Wstęp}
Tutaj będzie wstęp.

\chapter{Wprowadzenie teoretyczne}

W tej części zostaną wprowadzone wszelkie potrzebne miary skuteczności modeli oraz modele predykcyjne, które zostaną wykorzystane do przeprowadzenia eksperymentu. 

\section{Miary skuteczności modeli}

\subsection{Macierz pomyłek}


\begin{table}
	\begin{center}
		\makegapedcells
		\begin{tabular}{cc|cc}
			\multicolumn{2}{c}{}     &   \multicolumn{2}{c}{Predykcja} \\
			&            &   Oszustwo &   Normalna     \\ 
			\cline{2-4}
			\multirow{2}{*}{\rotatebox[origin=c]{90}{Prawda}} & Oszustwo   & TP         & FN              \\
			& Normalna   & FP         & TN              \\ 
			\cline{2-4}
		\end{tabular}
	\end{center}
	\caption{Macierz pomyłek}
	\label{macierz-pomylek}
\end{table}


Na podstawie podanej macierzy pomyłek w tabeli \ref{macierz-pomylek} definiujemy następujące miary skuteczności modeli:

$$ \text{Skuteczność} = \frac{TP + TN}{TP + FP + FN + TN} $$
$$ \text{Precyzja} = \frac{TP}{TP + FP} $$
$$ \text{Czułość}= \frac{TP}{TP + FN} $$
$$ \text{F1 Score} = 2 \cdot \frac{\text{Precyzja} \cdot \text{Czułość}}{\text{Precyzja} + \text{Czułość}} $$

\subsection{Metryki wrażliwe na koszt}

\section{Standardowe modele}

\subsection{Regresja logistyczna}

\subsection{Drzewo decyzyjne}

\subsection{Las losowy}

\subsection{XGBoost}

\section{Cost Sensitive Training}

\subsection{Regresja logistyczna wrażliwa na koszt}

\subsection{Drzewo decyzyjne wrażliwe na koszt}

\section{Cost Dependent Classification}

\subsection{Optymalizacja progu}

\subsection{Bayesian Minimum Risk}

\chapter{Eksperyment}

\chapter{Rezultaty}

\chapter{Podsumowanie}

\chapter{Dalsze prace}


\end{document}